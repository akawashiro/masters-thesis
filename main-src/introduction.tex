% !TEX root = ../main.tex

\section{Introduction}
\label{sec:intro}

\subsection{Multi-stage Programming and MetaOCaml}

% What is multi-stage programming?

Multi-stage programming makes it easier for programmers to implement
generation and execution of code at run time by providing language
constructs for composing and running pieces of code as first-class
values. A promising application of multi-stage programming is
(run-time) code specialization, which generates program code
specialized to partial inputs to the program and such applications are
studied in the literature~\cite{8384206,mainland2012metahaskell,taha2007gentle}.

% MetaOCaml

MetaOCaml~\cite{calcagno2003implementing,oleg2014} is an extension of
OCaml\footnote{\url{http://ocaml.org}} with special constructs for multi-stage
programming, including brackets and escape, which are (hygienic)
quasi-quotation, and \texttt{run}, which is similar to \texttt{eval} in Lisp,
and cross-stage persistence (CSP)~\cite{MetaML}.  Programmers can easily write
code generators by using these features.  Moreover, MetaOCaml is equipped with
a powerful type system for safe code generation and execution.  The notion of
code types is introduced to prevent code values that represent ill-typed
expressions from being generated.  For example, a quotation of expression
\texttt{1 + 1} is given type \texttt{int code} and a code-generating function,
which takes a code value \(c\) as an argument and returns \(c \texttt{ + } c\),
is given type \texttt{int code -> int code} so that it cannot be applied to,
say, a quotation of \texttt{"Hello"}, which is given type \texttt{string code}.
Ensuring safety for \verb|run| is more challenging because code types by
themselves do not guarantee that the execution of code values never results in
unbound variable errors.  Taha and Nielsen~\cite{taha2003environment}
introduced the notion of environment classifiers to address the problem,
developed a type system to ensure not only type-safe composition but also
type-safe execution of code values, and proved a type soundness theorem (for a
formal calculus \(\lambda^\alpha\) modeling a pure subset of MetaOCaml).

% The problem of existing multi-stage programming type system

However, the type system, which is based on the Hindley--Milner
polymorphism~\cite{Milner78JCSS}, is not strong enough to guarantee
invariant beyond simple types.  For example, Kiselyov~\cite{8384206}
demonstrates specialization of vector/matrix computation with respect
to the sizes of vectors and matrices in MetaOCaml but the type system
of MetaOCaml cannot prevent such specialized functions from being applied to
vectors and matrices of different sizes.

\subsection{Multi-stage Programming with Dependent Types}

% 依存型とは何か?

One natural idea to address this problem is the introduction of dependent types
to express the size of data structures in static types~\cite{Xi98}.  For
example, we could declare vector types indexed by the size of vectors as
follows.
\begin{verbatim}
    Vector :: Int -> *
\end{verbatim}
\verb|Vector| is a type constructor that takes an integer (which
represents the length of vectors): for example, \verb|Vector 3| is the
type for vectors whose lengths are 3.  Then, our hope is to specialize
vector/matrix functions with respect to their size and get a piece of
function code, whose type respects the given size, \emph{provided at
  specialization time}.  For example, we would like to specialize a
function to add two vectors with respect to the size of vectors, that
is, to implement a code generator that takes a (nonnegative) integer $n$ as an
input and generates a piece of function code of type
\verb|(Vector |$n$\verb| -> Vector |$n$\verb| -> Vector |$n$\verb|) code|.

\subsection{Our Work}
In this paper, we develop a new multi-stage calculus \LMD by extending
the existing multi-stage calculus \LTP\cite{Hanada2014} with dependent
types and study its properties.  We base our work on \LTP, in which
the four multi-stage constructs are handled slightly differently from
MetaOCaml, because its type system and semantics are arguably simpler
than \(\lambda^\alpha\)~\cite{taha2003environment}, which formalizes
the design of MetaOCaml more faithfully.  Dependent types are based on
\LLF~\cite{attapl}, which has one of the simplest forms of dependent
types.  Our technical contributions are summarized as follows:
\begin{itemize}
    \item We give a formal definition of \LMD with its syntax, type system and
        two kinds of reduction: full reduction, allowing reduction of any redex,
        including one under $\lambda$-abstraction and quotation, and staged reduction, a
        small-step call-by-value operational semantics that is closer to the intended
        multi-stage implementation.
    \item We show preservation, strong normalization, and confluence for
        full reduction; and show unique decomposition (and progress as its
        corollary) for staged reduction.
    \item We give algorithmic typing for \LMD which is essential to implement a
        type checker and show the equivalence between algorithmic typing and
        normal typing.
\end{itemize}
The combination of multi-stage programming and dependent types has been
discussed by Pasalic, Taha, and Sheard~\cite{pasalic2002tagless} and Brady and
Hammond~\cite{brady2006dependently} but, to our knowledge, our work is a first
formal calculus of full-spectrum dependently typed multi-stage programming with
all the key constructs mentioned above.

\subsubsection{Organization of the Thesis.}

The organization of this paper is as follows.
Section~\ref{sec:informal-overview} gives an informal overview of
\LMD. Section~\ref{sec:formal-definition} defines \LMD and
Section~\ref{sec:properties} shows properties of \LMD.
Section~\ref{sec:related-work} discusses related work and Section
\ref{sec:conclusion} concludes the paper with discussion of future
work.
